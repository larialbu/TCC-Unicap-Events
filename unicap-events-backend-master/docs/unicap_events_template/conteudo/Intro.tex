\chapter{Introdução}

\section{Contextualização e Motivação}

O Instituto Humanitas da Universidade Católica de Pernambuco tem sido um ponto de referência crucial para o desenvolvimento acadêmico e cultural da comunidade universitária. Como parte integrante deste ambiente enriquecedor, reconhecemos a importância de uma plataforma de eventos eficiente e acessível para manter a comunidade informada e envolvida em todas as atividades promovidas pelo instituto. \\

Ao longo dos anos, a plataforma existente tem desempenhado um papel fundamental na administração dos eventos, palestras e outras iniciativas acadêmicas e culturais. No entanto, sua estrutura desatualizada tem enfrentado obstáculos significativos para a plena realização de seu potencial. Portanto, surge a necessidade premente de criar uma nova plataforma que não apenas resolva as deficiências da anterior, mas também ofereça uma experiência mais intuitiva, moderna e atraente para os usuários.

\section{Problemática}
A plataforma de eventos existente estava defasada em termos de design, usabilidade e funcionalidades. Sua interface desatualizada e falta de recursos adequados estavam comprometendo a eficiência na organização e divulgação dos eventos. Além disso, a experiência do usuário era prejudicada, o que impactava negativamente na participação e engajamento da comunidade acadêmica. \\

Em resumo, os principais problemas encontrados consistem de:

\begin{itemize}
    \item Limitações de design e usabilidade, afetando a experiência do usuário.
    \item Falta de recursos modernos para gerenciamento eficiente de eventos.
\end{itemize} 

\section{Objetivos}

\subsection{Objetivo Geral}

Implementar uma nova plataforma de eventos para o Instituto Humanitas, que seja moderna, intuitiva e eficiente, proporcionando uma experiência agradável e funcional para os usuários, desde a organização até a participação nos eventos.

\subsection{Objetivos Específicos}

\begin{itemize}
  \item Desenvolver uma interface visualmente atrativa e responsiva, que seja acessível em dispositivos móveis e desktops.
  \item Implementar um sistema de gerenciamento de eventos completo, permitindo o cadastro, edição e acompanhamento detalhado das atividades.
  \item Introduzir funcionalidades de notificação e divulgação de eventos para manter a comunidade informada.
\end{itemize}