\chapter{Trabalhos Relacionados}

Ao adentrar no campo de gestão de eventos acadêmicos de uma universidade, diversas iniciativas similares fornecem insights valiosos para o desenvolvimento do nosso próprio aplicativo. Estas referências podem ser fundamentais para aprimorar a experiência dos participantes e otimizar a organização dos eventos. A seguir, apresentamos alguns trabalhos relacionados que servem como pontos de partida para nossa proposta: \\

    1. O avanço tecnológico na área de gestão de informações, conforme abordado por Lira et al. (2004), destaca a importância de acompanhar as transformações tecnológicas para otimizar processos e melhorar a qualidade dos serviços oferecidos. Da mesma forma, podemos integrar soluções digitais em nossa plataforma de eventos para aprimorar a organização, o acesso e o gerenciamento dos eventos acadêmicos. \\
    
    2. O Even3, uma plataforma digital dedicada à gestão de eventos acadêmicos, tem sido amplamente adotado em instituições de ensino superior. Similarmente ao que almejamos, o Even3 oferece funcionalidades como cadastro de eventos, submissão de trabalhos, inscrições de participantes e emissão de certificados. Além disso, a plataforma facilita a interação entre os participantes, promovendo uma experiência mais enriquecedora. A experiência acumulada pelo Even3 pode ser uma fonte valiosa de inspiração e orientação para a nossa própria aplicação. \\
    
    3. O Eventbrite é outra referência relevante, sendo uma plataforma versátil para a organização de eventos em diversos contextos. Embora seja mais conhecido por eventos culturais e comerciais, o Eventbrite também é utilizado em eventos acadêmicos. Ele oferece recursos como criação de eventos, venda de ingressos, gerenciamento de participantes e promoção através de redes sociais. Podemos explorar as funcionalidades do Eventbrite e adaptá-las às necessidades específicas dos eventos acadêmicos do Instituto Humanitas.

    